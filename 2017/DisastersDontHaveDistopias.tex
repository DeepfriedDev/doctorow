\documentclass{article}

\title{Disasters Don’t Have to End in Dystopias}
\date{04/05/2017}
 

\begin{document}

\maketitle


My new novel, Walkaway, is about a world where the super­rich create immortal life-forms (corporations) so effective at automating away labor that the rest of us become surplus resources. The ensuing battle—over whether humanity will finally, permanently speciate into elite transhumans and teeming, climate-wracked refugees—triggers slaughter and persecution. It’s a utopian novel.

The difference between utopia and dystopia isn’t how well everything runs. It’s about what happens when everything fails. Here in the nonfictional, disastrous world, we’re about to find out which one we live in.

Since Thomas More, utopian projects have focused on describing the perfect state and mapping the route to it. But that’s not an ideology, that’s a daydream. The most perfect society will exist in an imperfect universe, one where the second law of thermodynamics means that everything needs constant winding up and fixing and adjusting. Even if your utopia has tight-as-hell service routines, it’s at risk of being smashed by less-well-maintained hazards: passing aster­oids, feckless neighboring states, mutating pathogens. If your utopia works well in theory but degenerates into an orgy of cannibalistic violence the first time the lights go out, it is not actually a utopia.

I took inspiration from some of science fiction’s most daring utopias. In Kim Stanley Robinson’s Pacific Edge—easily the most uplifting book in my collection—a seemingly petty squabble over zoning for an office park is a microcosm for all the challenges that go into creating and maintaining a peaceful, cooperative society. Ada Palmer’s 2016 fiction debut, Too Like the Lightning, is a utopia only a historian could have written: a multi­polar, authoritarian society where the quality of life is assured by a mix of rigid social convention, high tech federalism, and something almost like feudalism.

The great problem in Walkaway (as in those novels) isn’t the exogenous shocks but rather humanity itself. It’s the challenge of getting walkaways—the 99 percent who’ve taken their leave of society and thrive by cleverly harvesting its exhaust stream—to help one another despite the prepper instincts that whisper, “The disaster will only spare so many of its victims, so you’d better save space on any handy lifeboats, just in case you get a chance to rescue one of your own.” That whispering voice is the background hum of a society where my gain is your loss and everything I have is something you don’t—a world where material abundance is perverted by ungainly and unstable wealth distribution, so everyone has to worry about coming up short.

(Recall that half the seats on many of the Titanic’s lifeboats were empty. Some toxic combination of panic and unco­operativeness drove those who made it to safety to leave those benches half-filled, even as more than 1,500 passen­gers drowned around them.)

Here’s how you can recognize a dystopia: It’s a science fiction story in which disaster is followed by brutal, mindless violence. Here’s how you make a dystopia: Convince people that when disaster strikes, their neighbors are their enemies, not their mutual saviors and responsibilities. The belief that when the lights go out, your neighbors will come over with a shotgun—rather than the contents of their freezer so you can have a barbecue before it all spoils—isn’t just a self-fulfilling prophecy, it’s a weaponized narrative. The belief in the barely restrained predatory nature of the people around you is the cause of dystopia, the belief that turns mere crises into catastrophes.

Stories of futures in which disaster strikes and we rise to the occasion are a vaccine against the virus of mistrust. Our disaster recovery is always fastest and smoothest when we work together, when every seat on every lifeboat is taken. Stories in which the breakdown of technology means the breakdown of civilization are a vile libel on humanity itself. It’s not that some people aren’t greedy all the time (or that all of us aren’t greedy some of the time). It’s about whether it’s normal to act on our better natures or whether our worst instincts are so intrinsic to our humanity that you can’t be held responsible for surrendering to them.

Our technology has revealed much of human nobility and cruelty. It’s given us global troll armies, to be sure—but also communities of mutual aid, support across vast distances, mobs of good people effecting one internet-­based barn­raising after another. Science fiction stories about the net “predicted” both, but the best science fiction does some­thing much more interesting than prediction: It inspires. That science fiction tells us better nations are ours to build and lets us dream vividly of what it might be like to live in those nations.

Last year was full of disasters, and 2017 is shaping up to be more disastrous still—nothing we do will change that. Disasters are part of the universe’s great unwinding, the fundamental perversity of inanimate matter’s remorseless disordering. But whether those disasters are dystopias? That’s for us to decide, and the deciding factor might just be the stories we tell ourselves.


\end{document}